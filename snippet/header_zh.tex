\documentclass[11pt,a4paper,twoside]{book}

\usepackage{fontspec}
\setmainfont{Times New Roman}
\setsansfont{Arial}
\setmonofont{Courier New}

\usepackage[BoldFont,SlantFont,CJKchecksingle,CJKnumber]{xeCJK}
\setCJKmainfont[BoldFont={Adobe Heiti Std},ItalicFont={Adobe Kaiti Std}]{Adobe Song Std}
\setCJKsansfont{Adobe Heiti Std}
\setCJKmonofont{Adobe Fangsong Std}
\punctstyle{hangmobanjiao}

\defaultfontfeatures{Mapping=tex-text}
\usepackage{xunicode}
\usepackage{xltxtra}

\XeTeXlinebreaklocale "zh"
\XeTeXlinebreakskip = 0pt plus 1pt minus 0.1pt

%首行缩进
%\usepackage{indentfirst}
\makeatletter
\let\@afterindentfalse\@afterindenttrue
\@afterindenttrue
\makeatother
\setlength{\parindent}{2em}

\linespread{1.2}

\usepackage[a4paper,top=2.5cm,bottom=1.5cm,left=2cm,right=1.5cm,marginparwidth=4.7cm,marginparsep=0.3cm]{geometry}

%在每一章的开头列出section的列表
\usepackage{minitoc}
\setcounter{minitocdepth}{1}
\renewcommand{\mtctitle}{目录}

%修改目录中的章节格式
\usepackage{titletoc}
\titlecontents{chapter}[0pt]{\vspace{0.3\baselineskip}\bfseries}{第 \thecontentslabel 章\quad}{}{\hfill\contentspage}
\titlecontents{section}[2em]{\vspace{0.05\baselineskip}}{\thecontentslabel\quad}{}{\hspace{.5em}\titlerule*[8pt]{$\cdot$}\contentspage}
\titlecontents{subsection}[4em]{\vspace{0.02\baselineskip}}{\thecontentslabel\quad}{}{\hspace{.5em}\titlerule*[8pt]{$\cdot$}\contentspage}
\titlecontents{figure}[0pt]{\vspace{0.05\baselineskip}}{\thecontentslabel\quad}{}{\hspace{.5em}\titlerule*[8pt]{$\cdot$}\contentspage}
\titlecontents{table}[0pt]{\vspace{0.05\baselineskip}}{\thecontentslabel\quad}{}{\hspace{.5em}\titlerule*[8pt]{$\cdot$}\contentspage}

%整段缩进
\usepackage{changepage}

%图文混排
%\usepackage{picins}

%在footnote中使用verb
\usepackage{fancyvrb}
%在section中使用\verb
\usepackage{cprotect}

%文本框
%\usepackage{fancybox}

\usepackage{fancyhdr}
\pagestyle{fancy}
\renewcommand{\chaptermark}[1]{\markboth{\small 第 \thechapter 章\quad #1}{}}
\renewcommand{\sectionmark}[1]{\markright{\small \thesection \quad #1}{}}
\fancyhf{}
\fancyhead[ER]{\leftmark}
\fancyhead[OL]{\rightmark}
\fancyhead[EL,OR]{$\cdot$ \thepage \ $\cdot$}
\renewcommand{\headrulewidth}{0.5pt}

%单面模式
%\newpagestyle{main}{
%\sethead[$\cdot$~\thepage~$\cdot$][][\chaptername\quad\chaptertitle]{\small\S\,\thesection\quad\sectiontitle}{}{$\cdot$~\thepage~$\cdot$}
%\setfoot{}{}{}\headrule}
%\pagestyle{main}

\usepackage{titlesec}
\titleformat{\chapter}{\centering\LARGE\bfseries}{第 \thechapter 章}{1em}{}

%附录
\usepackage{appendix}

% \usepackage[dvipsnames,tables]{xcolor}
\usepackage{xcolor}
\usepackage{graphicx}
\graphicspath{{figures/}}
%\usepackage[xetex,bookmarksnumbered=true,bookmarksopen=true,pdfborder=1,breaklinks,colorlinks,linkcolor=blue,urlcolor=blue,citecolor=blue]{hyperref}
\usepackage[xetex,bookmarksnumbered=true,bookmarksopen=true,pdfborder=1,breaklinks,colorlinks]{hyperref}
% \usepackage[xetex,bookmarksnumbered=true,bookmarksopen=true,pdfborder=1,breaklinks]{hyperref}

\def\chapterautorefname~#1\null{第{#1}章\null}
\def\sectionautorefname~#1\null{第{#1}节\null}
\def\subsectionautorefname~#1\null{第{#1}小节\null}
\def\subsubsectionautorefname~#1\null{第{#1}小节\null}
\def\figureautorefname{图}
\def\tableautorefname{表}
\def\Appendixautorefname~#1\null{附录{#1}\null}
\def\lstlistingautorefname{例}

\renewcommand{\today}{\number\year 年 \number\month 月 \number\day 日}
\renewcommand{\contentsname}{目\quad 录}
\renewcommand{\listfigurename}{图\quad 片}
\renewcommand{\listtablename}{表\quad 格}
\renewcommand{\figurename}{图}
\renewcommand{\tablename}{表}
%\renewcommand{\lstlistingname}{例}
\renewcommand{\indexname}{索\quad 引}
%\renewcommand{\figureautorefname}{图}
%\renewcommand{\tableautorefname}{表}
\renewcommand{\footnoteautorefname}{脚注}
\renewcommand{\itemautorefname}{项}

%使表格可以跨页
\usepackage{booktabs,tabu,longtable}
%调整表头和表格之间的间距
\setlength\belowcaptionskip{0.2em}
%调整表格行高
\renewcommand{\arraystretch}{0.8}

%调整列表间及其上下的间距
\usepackage{enumitem}
\setlist{nosep}

%设置颜色的快捷命令
\newcommand{\red}{\textcolor{red}}
\newcommand{\gray}{\textcolor{gray}}
\newcommand{\black}{\textcolor{black}}

%罗马数字
\makeatletter
\newcommand{\rmnum}[1]{\romannumeral #1}
\newcommand{\Rmnum}[1]{\expandafter\@slowromancap\romannumeral #1@}
\makeatother

%插入源代码
\usepackage{listings}
%The header name for the list of listings: Listings
\renewcommand{\lstlistlistingname}{程\quad 序}
%The caption label for listings: Listing
\renewcommand{\lstlistingname}{例}
\contentsuse{lstlisting}{lol}
\titlecontents{lstlisting}[0pt]{\vspace{0.05\baselineskip}}{}{\thecontentslabel\quad}{\hspace{.5em}\titlerule*[8pt]{$\cdot$}\contentspage}
\lstset{
  language=Perl,
  %basicstyle=\normalsize\tt,
  basicstyle=\small\tt,
  frame=l,
  numbers=left,
  numberstyle=\footnotesize,
  showstringspaces=false,
  %breaklines=true,
  breaklines=false,
  breakatwhitespace=false,
}
\usepackage{caption}
\captionsetup[lstlisting]{labelformat=empty,labelsep=none,format=plain,font=bf}
%\captionsetup[lstlisting]{labelformat=empty,labelsep=none,textformat=empty,skip=-20pt}

%suppresses page numbers and headings from appearing on empty pages.
%remove headers and footers for pages between chapters
\usepackage{emptypage}

\usepackage{mdframed}
