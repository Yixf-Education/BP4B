\thispagestyle{empty}
该书向只有很少、甚至没有编程经验的生物学家展示了如何使用Perl这一理想的编程语言进行生物学数据分析。每一章都集中解决特定的问题或者问题集,所以当你读完本书后,你会对Perl的基础知识有一个深刻的理解,收集到解析BLAST和GenBank等任务的程序,同时习得处理更加高级的生物信息学问题的技能。

\vfill
%\doublebox{
%{\centering \large \bf 译文版权声明}
\begin{mdframed}[linecolor=red,linewidth=2pt,backgroundcolor=red!10]
\begin{center}
{\LARGE \bf 译文版权声明}
\end{center}

\vspace{1em}

\begin{itemize}
  \item 本书的中文翻译(含封面)未得到原作者和原出版社的许可,中译本的翻译错误与原作者无关!
  \item 本书的中译本封面由原书封面修改而成,仅供该中译本使用!
  \item 本书的中译本初衷是作为天津医科大学、生物医学工程与技术学院、生物信息学专业、《分子生物计算》课程的教材。
  \item 本书的中译本仅供参考学习只用,严禁贩卖、流通,后果自负!
  \item 本书的翻译仍处于草稿阶段,疏漏、错误之处在所难免,欢迎读者予以指正。
  \item 本书的中译本解释权归译者所有,如有任何疑问,请发邮件至 \href{mailto:yixfbio@gmail.com}{yixfbio@gmail.com} 与译者本人联系。
\end{itemize}
\begin{flushright}
伊现富\\
2015年8月14日\\
天医-生医-208
\end{flushright}

\begin{center}
(版本:\today)
\end{center}
\end{mdframed}
%}

\newpage
\null
\thispagestyle{empty}
\newpage
