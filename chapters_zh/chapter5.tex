\chapter{基序和循环}
\label{chap:chapter5}
\minitoc

本章将在\autoref{chap:chapter4}的基础上进一步介绍Perl语言的基础知识。到本章结束的时候,你将学会:

\begin{itemize}
  \item 在DNA或蛋白质中查找基序
  \item 通过键盘与用户进行交互
  \item 把数据写入文件
  \item 使用循环
  \item 使用基本的正则表达式
  \item 根据条件测试的结果采取不同的行动
  \item 通过字符串和数组的操作对序列数据进行细致的处理
\end{itemize}

所有这些主题,加上在\autoref{chap:chapter4}学习的知识,将使你能够编写出实用的生物信息学程序。在本章中,你将学习编写一个在序列数据中查找基序的程序。

\section{流程控制}
\textit{流程控制}指的就是程序中语句执行的顺序。除非明确指明不按顺序执行,否则程序将从最顶端的第一个语句开始,顺序执行到最底端的最后一个语句。有两种方式可以告诉程序不按照顺序执行:条件语句和循环。\textit{条件语句}只会在条件测试成功的前提下执行相应的语句,否则就会直接跳过这些语句。\textit{循环}会一直重复那些语句,直到相应的测试失败为止。

\subsection{条件语句}
让我们再看一下\textit{open}语句吧。回忆一下,当你试图打开一个并不存在的文件时,你会看到错误信息。在你尝试打开文件之前,可以对文件是否存在进行明确的测试。事实上,类似的测试都是计算机语言最强大的特性之一。\textit{if}、\textit{if-else}和\textit{unless}条件语句就是Perl语言中用来进行类似测试的三个语句。

这类语句的最主要特性就是可以对条件进行测试。条件测试的结果可以是\verb|真(true)|或者是\verb|假(false)|。如果测试结果为\verb|真|,那么后面的语句就会被执行;与之相反,如果测试结果为\verb|假|,这些语句就会被跳过而不执行。

那么,“什么是真呢”?对于这个问题,不同计算机语言的回答会稍有不同。

本节包含了一些演示Perl条件语句的实例,其中的真-假条件测试就是看看两个数字是否相等。注意,因为一个等号\verb|=|是用来对变量进行赋值的,所以两个数字的相等要用两个等号\verb|==|来表示。

\begin{adjustwidth}{2cm}{2cm}
  \parpic[l]{
  \includegraphics[width=1cm]{warning.png}
}
\noindent
=表示赋值、==表示数字相等,这两者非常容易让人混淆,可是说是编程中最常见的“bug”了,所以一定要小心奥!
\end{adjustwidth}

下面这个例子演示了条件测试的结果是\verb|真|还是\verb|假|。平时你很少会用到这么简单的测试;通常,你会测试那些预先并不知道结果的值,比如读入变量的值,或者函数调用的返回值。

条件测试为\verb|真|的\verb|if|语句:

\begin{lstlisting}
if( 1 == 1 ) {
  print "1 equals 1\n\n";
}
\end{lstlisting}

它会输出:

\begin{lstlisting}
1 equals 1
\end{lstlisting}

我们进行的测试是\verb|1 == 1|,用口语来说就是,“1是否等于1”。1确实等于1,所以条件测试的结果为\verb|真|,因此和\verb|if|语句相关联的语句就会被执行,信息被打印了出来。

你也可以这样写:

\begin{lstlisting}
if( 1 ) {
  print "1 evaluates to true\n\n";
}
\end{lstlisting}

它会输出:

\begin{lstlisting}
1 evaluates to true
\end{lstlisting}

条件测试为\verb|假|的\verb|if|语句:

\begin{lstlisting}
if( 1 == 0 ) {
  print "1 equals 0\n\n";
}
\end{lstlisting}

它没有任何输出!我们进行的测试是\verb|1 == 0|,用口语来说就是,“1是否等于0”。1当然不等于0了,所以条件测试的结果为\verb|假|,因此和\verb|if|语句相关联的语句就不会被执行,不会输出任何信息。

你也可写成:

\begin{lstlisting}
if( 0  ) {
  print "0 evaluates to true\n\n";
}
\end{lstlisting}

因为0的测试结果为\verb|假|,所以它不会产生任何输出,因此和\verb|if|语句相关联的语句会完全被跳过。

还有一种编写短小的\verb|if|语句的方法,它模拟的是英语中if的用法。在英语中,你可以说,“If you build it, they will come”,或者,“They will come if you build it”,这两种说法是完全等同的。Perl也不甘示弱,它也允许你把\verb|if|放在对应的动作之后:

\begin{lstlisting}
print "1 equals 1\n\n" if (1 == 1);
\end{lstlisting}

它和本节中的第一个例子做的事情是完全一样的,输出:

\begin{lstlisting}
1 equals 1
\end{lstlisting}

现在,让我们看一下条件测试为\verb|真|的\verb|if-else|语句:

\begin{lstlisting}
if( 1 == 1 ) {
  print "1 equals 1\n\n";
} else {
  print "1 does not equal 1\n\n";
}
\end{lstlisting}

它会输出:

\begin{lstlisting}
1 equals 1
\end{lstlisting}

当测试结果为\verb|真|时,\verb|if-else|会做某件事情,当测试结果为\verb|假|时,它则会另一件事情。这是条件测试为\verb|假|的\verb|if-else|语句:

\begin{lstlisting}
if( 1 == 0 ) {
  print "1 equals 0\n\n";
} else {
  print "1 does not equal 0\n\n";
}
\end{lstlisting}

它会输出:

\begin{lstlisting}
1 does not equal 0
\end{lstlisting}

最后的例子是\verb|unless|,它和\verb|if|完全相反。它就像英语中的“unless”那样来使用:比如,“Unless you study Russian literature, you are ignorant of Chekov”\footnote{译者注:除非你研究俄国文学,否则你不会对契诃夫有所了解。}。如果测试结果为\verb|真|,它不会采取任何行动;而当测试结果为\verb|假|时,相应的语句就会被执行。如果“you study Russian literature”是\verb|假|的,那么“you are ignorant of Chekov”。\footnote{译者注:如果“你研究俄国文学”是\verb|假|的,那么“你对契诃夫毫无了解”。}

\begin{lstlisting}
unless( 1 == 0 ) {
  print "1 does not equal 0\n\n"\nonumber\\
}
\end{lstlisting}

它会输出:

\begin{lstlisting}
1 does not equal 0
\end{lstlisting}

\subsubsection{条件测试和括号成对}
对这些语句和它们的条件测试,还有两点注释:

第一点,在这些语句的条件部分,有许多测试可以使用。除了在前述实例中使用的数字\textit{相等}\verb|==|外,你还可以对不等\verb|!=|、大于\verb|>|和小于\verb|<|等进行测试。

与之类似,你也可以使用\textit{eq}操作符对字符串相等进行测试:如果两个字符串是一样的,测试结果就是\verb|真|。也有一些文件测试操作符,允许你对文件进行相应的测试:是否存在,是否为空,是否设置了特定的权限,等等(参看\autoref{chap:chapterab})。另一个比较常用的是对变量名进行测试:如果变量存储的是0,那它就被认为是\verb|假|的;任何其他的数字都被认为是\verb|真|的。如果变量中有非空的字符串,那它就是\verb|真|的;空的字符串用""或者''来表示,它是\verb|假|的。

第二点,注意条件测试之后的语句都被包裹在了成对的大括号内。用大括号括起来的语句叫做块,这在Perl中非常常见。\footnote{虽然这看起来有些奇怪,但块中的最后一个语句并不需要用分号来结尾。}括号成对出现是最普遍的编程特性,比如成对的小括号、中括号、尖括号和大括号:\verb|()|、\verb|[]|、\verb|<>|和\verb|{}|左右括号的数目相等,且它们都出现在正确的位置,这对于Perl程序正常运行来说是至关重要的。

成对使用括号是很难做到的,所以,当你因为少写了括号而导致程序报错时,请不要大惊小怪。这是一个比较常见的语法错误,你需要查阅代码找到缺少括号的地方并把它添加上。随着代码越来越复杂,要找到成对括号出错的地方并对它进行修正会更加具有挑战性。即使括号都在正确的位置,当你阅读代码时,要分清哪些语句是一组也是非常困难的。当然你也可以采取一定的措施来避免这样的问题:每一行代码做的事情不要太多,使用缩进让代码块更加突出明显一些(参看\autoref{sect:section5.2})。\footnote{有些文本编辑器可以帮你找到配对的括号(举个例子,在vi编辑器中,在一个括号上键入百分号\%就会让光标跳跃到与之配对的括号上去)。} 

回过头来再看看条件语句。就像\autoref{exam:example5.1}中演示的那样,\textit{if-else}还有\textit{if-elsif-else}这样的形式。第一个\textit{if}和\textit{elsifs}中的条件被交替测试,一旦测试结果为\verb|真|,相应的代码块就会被执行,并且剩余的条件测试也会被忽略掉。假设没有一个条件的测试结果为\verb|真|,如果存在\textit{else}代码块,那它就会被执行。\textit{else}代码块是可有可无的。

\textbf{例5-1:if-elsif-else}
\lstinputlisting[label=exam:example5.1]{./scripts/example5-1.pl}

注意\textit{else}代码块的\verb|print|语句中的\verb|\"|,它可以在双引号包裹起来的字符串中打印出双引号(")。反斜线告诉Perl,把后面紧跟的"当成引号本身,而不是把它当做字符串结束的标志。此外还请注意使用了\textit{eq}来检测字符串的相等。

\autoref{exam:example5.1}的输出:

\begin{lstlisting}
MNIDDKL--the magic word!
\end{lstlisting}

\subsection{循环}
\textit{循环}允许你重复执行被成对大括号包裹起来的语句块。在Perl中有多种方法实现循环:\textit{while}循环、\textit{for}循环、\textit{foreach}循环等等。\autoref{exam:example5.2}(来源于\autoref{chap:chapter4})演示了如何使用\textit{while}循环从一个文件读取蛋白质序列数据。

\textbf{例5-2:从文件中读取蛋白质序列数据,第四次尝试}
\lstinputlisting[label=exam:example5.2]{./scripts/example5-2.pl}

下面是\autoref{exam:example5.2}的输出:

\begin{lstlisting}
  ######  Here is the next line of the file:
MNIDDKLEGLFLKCGGIDEMQSSRTMVVMGGVSGQSTVSGELQD
  ######  Here is the next line of the file:
SVLQDRSMPHQEILAADEVLQESEMRQQDMISHDELMVHEETVKNDEEQMETHERLPQ
  ######  Here is the next line of the file:
GLQYALNVPISVKQEITFTDVSEQLMRDKKQIR
\end{lstlisting}

在\textit{while}循环中,注意在循环至文件的下一行时\verb|$protein|变量是如何一次次被赋值的。在Perl中,赋值语句会返回赋予的值。此处,条件测试检测的是赋值语句是否成功得读取了另一行。如果有另一行被读入,就会发生赋值行为,条件测试就为\verb|真|,新的一行会被存储在\verb|$protein|变量中,而包含两个\verb|print|语句的代码块也会被执行。如果没有更多行了,赋值就会是未定义,条件测试就为\verb|假|,而程序也会跳过包含两个\verb|print|语句的代码块,退出\textit{while}循环,继续执行程序的后续部分(在这个例子中,就是\textit{close}和\textit{exit}函数)。

\subsubsection{open和unless}
\textit{open}是一个系统调用,因为要打开一个文件,Perl必须向操作系统提出请求。操作系统可以是Unix或Linux、Microsoft Windows或Apple Maintosh等的某个版本。文件被操作系统管理者,因此也只有操作系统能够访问它。

检查系统调用的成功与否是一个好习惯,尤其是当打开文件时更是如此。如果系统调用失败了,而你也没有对它进行检查,那么程序将继续运行,可能会尝试读取或写入一个你并没有打开过的文件。你应该总是检查这种失败,当文件无法打开时,要立即告知程序的使用者。当打开文件失败时,通常情况下你会希望立即退出程序,并尝试打开另一个文件。

在\autoref{exam:example5.2}中,\textit{open}系统调用是\textit{unless}条件测试的一部分。

\textit{unless}和\textit{if}相反。就像在你英语中你可以说“do the statements in the block if the condition is true”(“如果条件为真,就执行代码块中的语句”)一样,你也可以反过来说“do the statements in the block unless the condition is true”(“除非条件为真,否则就执行代码块中的语句”)\footnote{译者注:根据语境,此处应为:“do the statements in the block unless the condition is false”(“除非条件为假,否则就执行代码块中的语句”)。}。如果成功打开了文件,\textit{open}系统调用就会返回真值;在这个例子的\textit{unless}语句条件测试中,如果\textit{open}系统调用失败了,那么代码块中的语句就会被执行,程序会输出错误信息,随后退出。

总结一下,在Perl中,条件和循环都是比较简单的概念,学起来并不困难。它们是编程语言最强大的特性之一。条件可以使你的程序适应不同的状况,使用这种方式,就可以针对不同的输入采取不同的方案了。在人工智能的计算机程序中,它占了相当大的比重。循环充分利用了计算机的速度,利用循环,仅仅使用几行代码,你就可以处理大量的输入,或者对计算进行重复迭代与提炼。

\section{代码布局}
\label{sect:section5.2}
一旦你开始使用循环和条件语句,你就需要认真考虑格式的问题了。当格式化Perl代码时你有多种选择。对于\verb|while|循环中的\textit{if}语句,比较一下下面几种不同的格式化方法:

\textcolor{red}{\textit{格式A}}
\begin{adjustwidth}{1cm}{}
\begin{lstlisting}
while ( $alive ) {
  if ( $needs_nutrients ) {
    print "Cell needs nutrients\n";
  }
}
\end{lstlisting}
\end{adjustwidth}

\textcolor{red}{\textit{格式B}}
\begin{adjustwidth}{1cm}{}
\begin{lstlisting}
while ( $alive )
{
  if ( $needs_nutrients )
  {
    print "Cell needs nutrients\n";
  }
}
\end{lstlisting}
\end{adjustwidth}

\textcolor{red}{\textit{格式C}}
\begin{adjustwidth}{1cm}{}
\begin{lstlisting}
  while ( $alive )
    {
      if ( $needs_nutrients )
{
  print "Cell needs nutrients\n";
}
}
\end{lstlisting}
\end{adjustwidth}

\textcolor{red}{\textit{格式D}}
\begin{adjustwidth}{1cm}{}
\begin{lstlisting}
while($alive){if($needs_nutrients){print "Cell needs nutrients\n";}}
\end{lstlisting}
\end{adjustwidth}

对于Perl解释器来说,所有这些代码片段都是一样的,因为Perl并不依赖于每一行中语句的布局方式,它只关心句法元素的正确顺序。有些元素在它们中间需要空白(比如空格、制表符或者换行符),这样就可以把它们区别开了,但通常来说,对于你使用空白来布局代码的方式,Perl不会进行特别的限定。

格式A和B是布局代码最常见的方式。它们都使程序结构更加清晰,便于人们阅读。对于\textit{while}和\textit{if}这种附带着代码块的语句来说,注意观察是如何排列大括号、以及缩进代码块中的语句的。这种布局使得语句附带的代码块的范围更加清晰一些。(这对于长的、复杂的代码块是非常关键的。)代码块中的语句都进行了缩进,通常你使用制表符键或者四个或八个空格就可以实现这种缩进。(许多文本编辑器可以使你在敲击制表符键时输入空格,或者你可以对它们进行配置,使得用制表符键就可以输入四个、八个或者任意数目的空格。)使用这种方式,程序的总体结构变得更加清晰了,你可以轻而易举得看出那些语句是归类在循环或条件语句附带的代码块中的。就个人而言,尽管我也非常乐意使用格式B,但我更加喜欢格式A的布局。

格式C是格式化代码的反面教材。代码的流控制很不清晰,举个例子,很难看出\textit{print}语句是否在\textit{while}语句的代码块中。

格式D演示了,在对代码不进行任何格式化的情况下,即使对于这样一个简单的代码片段来说,要阅读它是多么的困难。

从Perl的手册也中可以找到Perl的风格指南,或者输入下面的命令也可以:

\begin{lstlisting}
perldoc perlstyle
\end{lstlisting}

对于如何编写可读性好的代码,Perl风格指南给出了不少的建议和意见。然而,这些并不是规则,你可以使用在经过众多格式化练习后找到的最适合你自己的代码布局方案。

\section{查找基序}
在生物信息学中我们最常做的事情之一就是查找基序,即特定的DNA或蛋白质片段。这些基序可能是DNA的调控元件,也可能是已知在多个物种中保守的蛋白质片段。(PROSITE网站——\href{http://www.expasy.ch/prosite/}{http://www.expasy.ch/prosite/}——有关于蛋白质基序的更多信息。)

在生物学序列中你要查找的基序往往不是特定的一个序列,它们可能有一些变体——比如,某个位置上的碱基或者残基是什么并不重要。它们也有可能有不同的长度。但通常它们都可以用正则表达式来进行表征,在接下来的\autoref{exam:example5.3}、\autoref{chap:chapter9}以及本书中的许多地方,你都可以看到更多关于正则表达式的讨论。

Perl有一系列在字符串中进行查找的便利的特性,这同样使得Perl成为生物信息学领域中流行的语言。\autoref{exam:example5.3}对这种字符串搜索的能力进行了演示;它做的事情真的非常实用,许多类似的程序在生物学研究中一直被使用着。它做的事情包括:

\begin{itemize}
  \item 从文件中读入蛋白质序列数据
  \item 为了便于搜索,把所有的序列数据都放到一个字符串中
  \item 查找用户通过键盘键入的基序
\end{itemize}

\textbf{例5-3:查找基序}
\lstinputlisting[label=exam:example5.3]{./scripts/example5-3.pl}

这是\autoref{exam:example5.3}典型的一个输出:

\begin{lstlisting}
Please type the filename of the protein sequence data:
NM_021964fragment.pep
Enter a motif to search for: SVLQ
I found it!

Enter a motif to search for: jkl
I couldn't find it.

Enter a motif to search for: QDSV
I found it!

Enter a motif to search for: HERLPQGLQ
I found it!

Enter a motif to search for: 
I couldn't find it. 
\end{lstlisting}

就像你在结果中看到的那样,这个程序查找用户通过键盘键入的基序。利用这样一个程序,你就不用再在海量的数据中手工进行查找了。让计算机来做这样的工作,它比人工做的更快、更准确。

如果程序不仅报告它找到了基序,还能告诉我们在哪里找到了它,那就更好了。在\autoref{chap:chapter9}中你将看到如何来实现这个目标,那一章的一个练习要求你对这个程序进行修改,使得它能够报告基序的位置。

接下来的小节将对\autoref{exam:example5.3}中这些新的内容进行介绍与讨论:

\begin{itemize}
  \item 获取用户的键盘输入
  \item 把文件的所有行合并到一个标量变量中
  \item 正则表达式和字符组
  \item \verb|do-until|循环
  \item 模式匹配
\end{itemize}

\subsection{获取用户的键盘输入}
在\autoref{exam:example4.5}中你第一次遇到文件句柄。在\autoref{exam:example5.3}中(在\autoref{exam:example4.3}中也是这样),使用文件句柄和尖括号输入操作符从打开的文件中读入数据并保存到数组中,就像这样:

\begin{lstlisting}
@protein = <PROTEINFILE>;
\end{lstlisting}

Perl使用同样的语法获取用户的键盘输入。在\autoref{exam:example5.3}中,为了获取用户的键盘输入,使用了一个叫做STDIN(standard input(标准输入)的简写)的特殊的文件句柄,就像这行从用户键盘输入获取文件名的代码一样:

\begin{lstlisting}
$proteinfilename = <STDIN>;
\end{lstlisting}

文件句柄可以和一个文件相关联,也可以和用户回答程序问题时的键盘输入相关联。

就像这个代码片段一样,如果你用来存储输入的变量是一个以美元符号(\verb|$|)起始的标量变量,那么将只有一行被读入,这往往也是在这种情况下你希望的结果。

在\autoref{exam:example5.3}中,用户被要求输入一个包含蛋白质序列数据的文件的文件名。在通过这种方式得到文件名后,打开文件之前还需要一步操作。当用户键入文件名并通过Enter键(也叫做Return键)换行时,文件名和其末尾的换行符一起被保存进了变量中。这个换行符并不是文件名的一部分,所以在使用\textit{open}系统调用之前应该把它去除掉。Perl函数\textit{chomp}会去除掉字符串末尾的换行符newlines(或者它的表亲换行符linefeeds和回车符carriage returns)。(更加古老的\textit{chop}函数会删除最后一个字符,不管这个字符是什么;这会导致一些问题,所以\textit{chomp}被引入了进来,而它也总是被优先考虑使用的。)

所以这一个部分还有一点附带的知识:删除获取的用户键盘输入中的换行符。试着把\verb|chomp|函数注释掉,你会看到\verb|open|调用失败,因为并没有末尾包含换行符的文件名。(对于文件名中可以使用的字符,操作系统有自己的规定。)

\subsection{使用join把数组合并成标量}
\label{sect:section5.3.2}
常常可以发现,蛋白质序列数据被打断成了短的片段,每个片段有大约80个字符。原因很简单:当要把数据打印到纸张上或展示在屏幕上时,根据纸张或屏幕的空间需要把它打断成数行。你的数据被打断成了片段,但对你的Perl程序来说却多有不便。当你要查找一个基序,而它却被换行符分割开了,这会怎样呢?你的程序将无法找到这个基序。事实上,\autoref{exam:example5.3}中查找的一些基序确实被换行符分割开了。在Perl中,你可以使用\textit{join}函数来处理类似的片段数据。在\autoref{exam:example5.3}中,\textit{join}把存储在\verb|@protein|数组中的数据的所有行合并成一个单独的字符串,并把它保存在了新的标量变量\verb|$protein|中:

\begin{lstlisting}
$protein = join( '', @protein );
\end{lstlisting}

当合并数组时,你需要指定一个字符串,这个字符串会放在数组的各个元素之间。在这个例子中,你指定的是空字符串,它会放在输入文件的各行之间。空字符串就夹在成对的单引号''中(当然也可以使用双引号"")。


回忆一下\autoref{exam:example4.2},在那个例子中,我介绍了好几种串联两个DNA片段的方法。\textit{join}函数的作用与之类似,它取出数组元素的变量值并把它们串联成一个单独的标量值。回忆一下\autoref{exam:example4.2}中的下列语句,它是串联两个字符串的方法之一:

\begin{lstlisting}
$DNA3 = $DNA1 . $DNA2;
\end{lstlisting}

另一种实现同样串联的方法是使用\textit{join}函数:

\begin{lstlisting}
$DNA3 = join( "", ($DNA1, $DNA2) );
\end{lstlisting}

在这个例子中,我没有给出数组名,而是指定了一个标量元素的列表:

\begin{lstlisting}
($DNA1, $DNA2)
\end{lstlisting}

\subsection{do-until循环}
在\autoref{exam:example5.3}中有一种新的循环,这就是\verb|do-until|循环,它首先执行一个代码块,之后再进行条件检测。有时它会比常见的先测试、如果测试成功就执行代码块的策略更加方便。在这个例子中,你想提示用户,获取用户的输入,查找基序并报告查找结果。在重复这些动作之前,你进行条件测试来看看用户是否输入了一个空行。输入空行意味着用户没有更多的基序来查找了,所以你就退出循环。

\subsection{正则表达式}
\textit{正则表达式}使你可以轻松处理各种各样的字符串,比如DNA和蛋白质序列数据。正则表达式的强大之处就在于,如果想对字符串进行一些处理,你通常可以使用Perl的正则表达式来完成该任务。

有些正则表达式非常简单。举个例子,你可以把你想搜索的具体文字当成正则表达式来使用:如果我想在本书的文字中超找“bioinformatics”(“生物信息学”),我可以使用这样的正则表达式:

\begin{lstlisting}
/bioinformatics/
\end{lstlisting}

然而,有些正则表达式会更加复杂。在本节中,我会通过\autoref{exam:example5.3}来演示它们的使用。

\subsubsection{正则表达式和字符组}
正则表达式使用特定的类似于通配符的操作符来匹配一个或多个字符串。正则表达式可以非常简单——一个单词,它匹配单词本身,也可以非常复杂,可以匹配一个大的单词集合(甚至每一个单词!)。

在\autoref{exam:example5.3}中,当你把蛋白质序列数据合并到\verb|$protein|标量变量中后,你还需要删除掉换行符和其他所有不是序列数据的字符。这可能会包括行中的数字、注释、信息行或标题行,等等。在这个例子中,你需要删除掉换行符和空格或制表符等那些存在其中的非打印字符。下面这行\autoref{exam:example5.3}中的代码会删除空白:

\begin{lstlisting}
$protein =~ s/\s//g;
\end{lstlisting}

标量变量\verb|$protein|中的序列数据会被这个语句所改变。在\autoref{exam:example4.3}中你首次看到绑定操作符=~和替换函数\textit{s///},那时它们被用来把一个字符替换成另一个字符。而此处,它们的使用则有少许的不同。你把用\verb|\s|表示的空白字符组中的任意一个字符替换成了空字符串,第二个和第三个正斜线之间什么都没有表示的就是空字符串。换言之,你删除了空白字符组中的任意一个字符,这是对字符串进行的全局操作,这是通过语句末尾的\verb|g|来实现的。

\verb|\s|是众多元字符中的一个。你已经见过了\verb|\n|元字符。\verb|\s|元字符匹配空格、制表符、换行符、回车符和换页符中的任意一个。\verb|\s|也可以写成:

\begin{lstlisting}
[ \t\n\f\r]
\end{lstlisting}

这个表达式是字符组的一个例子,用中括号包裹起来。字符组匹配中括号里的字符中的任意一个。空格就用空格来表示,而其他空白字符则有自己的元字符:制表符用\verb|\t|表示,换行符用\verb|\n|表示,换页符用\verb|\f|表示,回车符用\verb|\r|表示。回车符使得下一个字符被写在行首,而换页符则会转换到下一行。两者结合起来实际上就是换行符的作用。

我提到的每一个\textit{s///}命令中都有一定形式的正则表达式,它位于前两个正斜线\verb|/|之间。在\textit{s/C/G/g}中处在该位置的是\verb|C|这样的单个字母。\verb|C|是有效正则表达式的一个例子。

在\autoref{exam:example5.3}中,也使用了正则表达式。下面这行代码:

\begin{lstlisting}
if ( $motif =~ /^\s*$/ ) {
\end{lstlisting}

用口语来说(用英语来说),就是检测\verb|$motif|变量中的空行。如果用户没有任何输入,或者只是输入了一些空白字符,空白字符用\verb|\s*|来表示,该匹配就会成功,程序则会退出。完整的正则表达式是:

\begin{lstlisting}
/^\s*$/
\end{lstlisting}

把它翻译过来就是:匹配这样的字符串,从开头(用\verb|^|表示)到结尾(用\verb|$|表示)只有零个或多个(用\verb|*|表示)空白字符(用\verb|\s|表示)。

如果觉得这些晦涩难懂,那就先别管它们了,很快你就会对这些术语越来越熟悉了。正则表达式是处理序列和其他基于文本的数据的很好的方法,而Perl则对正则表达式尤其擅长,使得它们易用、强大且灵活。\autoref{chap:chapteraa}中的许多参考资料都有关于正则表达式的相关内容,而在\autoref{chap:chapterab}中则有一个简短的总结。

\cprotect \subsubsection{使用\verb|=~|和正则表达式进行模式匹配}
真正的查找基序是\autoref{exam:example5.3}中的这一行: 

\begin{lstlisting}
if ( $protein =~ /$motif/ ) {
\end{lstlisting}

此处,绑定操作符\verb|=~|在蛋白质\verb|$protein|中查找存储在\verb|$motif|变量值中的正则表达式。通过这种特性,你可以把变量的值内插到字符串匹配中。(Perl字符串中的内插指的是把变量的值插入到字符串中,就像当你要把字符串串联起来时,在\autoref{exam:example4.2}中首次看到的那样)。真正的基序,也就是字符串变量\verb|$motif|的值,就是你的正则表达式。这些最简单的正则表达式就是由一些字符组成的字符串,就像\verb|AQQK|基序一样。

你也可以使用\autoref{exam:example5.3}来探索正则表达式的更多特性。你可以键入任意的正则表达式,在蛋白质中进行查找。参考正则表达式的文档,运行程序,去探索吧!这里是键入的正则表达式的一些例子:

\begin{itemize}
  \item 查找A、紧跟D或S、之后紧跟V的基序:
\begin{lstlisting}
Enter a motif to search for: A[DS]V
I couldn't find it.
\end{lstlisting}
  \item 查找K、N、零个或多个D、两个或更多个E(注意{2,}表示两个或更多)的基序:
\begin{lstlisting}
Enter a motif to search for: KND*E{2,}
I found it!
\end{lstlisting}
  \item 查找两个E、紧跟着任意字符、之后紧跟另外两个E的基序:
\begin{lstlisting}
Enter a motif to search for: EE.*EE
I found it!
\end{lstlisting}
\end{itemize}

在最后一个查找中,注意点号表示除换行符以外的任意字符,".*"代表零个或多个这样的字符。(如果你想匹配点号本身,需要使用反斜线对它进行转义。)

\section{计数核苷酸}
对于一段DNA序列,有许多信息你可能想知道。它是编码的还是非编码的?\footnote{编码DNA是编码蛋白质的DNA,也就是说,它是基因的一部分。在包括人类的许多生物中,大部分DNA是不编码的——它不是基因的一部分,也不编码蛋白质。在人类基因组中,大约98-99\%的DNA是非编码的。}它是否含有调控元件?它是否和其他已知的DNA相关,如果相关,是怎么相关呢?DNA中四种核苷酸的数目各是多少?事实上,在许多物种中,编码区域都有特定的核苷酸偏向性,所以,最后这个问题对于基因识别来说是非常重要的。此外,不同的物种有不同的核苷酸使用模式。所以对核苷酸进行计数不但有趣而且非常有用。

接下来的小节中有两个程序,\autoref{exam:example5.4}和\autoref{exam:example5.6},它们对DNA序列中每种核苷酸都进行计数。这两个程序展示了Perl的一些新的知识:

\begin{itemize}
  \item “拆解”字符串
  \item 查看字符串的特定位置
  \item 对数组进行迭代
  \item 对字符串的长度进行迭代
\end{itemize}

为了对DNA中的每一种核苷酸进行计数,你需要查看每一个碱基,看看它是什么,并且要为每一种碱基记录一个计数,一共四个计数。我们将通过两种方法来实现:

\begin{itemize}
  \item 把DNA拆解成单个碱基,存储到数组中,然后对数组进行迭代(换言之,一个接一个的对数组元素进行处理)
  \item 在计数时使用\textit{substr}Perl函数对DNA字符串的位置进行迭代
\end{itemize}

首先,让我们从该任务的伪代码开始。之后,我们会给出更加详细的伪代码,最终,编写出这两种方案的Perl程序。

下面的伪代码描述了基本的要求:

\begin{lstlisting}
for each base in the DNA
  if base is A
    count_of_A = count_of_A + 1
  if base is C
    count_of_C = count_of_C + 1
  if base is G
    count_of_G = count_of_G + 1
  if base is T
    count_of_T = count_of_T + 1
done

print count_of_A, count_of_C, count_of_G, count_of_T
\end{lstlisting}

就像你看到的一样,这是一个非常简单的想法,它模拟了你手工计数时的工作。(如果你想计算所有人类基因中碱基的相对频率,手工计数就不现实了,因为数量太过巨大,你将不得不使用类似的程序。这就是生物信息学。)现在,让我们看看如何用Perl来实现它吧。

\section{把字符串拆解成数组}
假设你打算把DNA字符串拆解成数组。所谓拆解,我指的是字符串中的每一个字母分离开来,就像把字符串分离成bit一样。换句话说,DNA字符串中表示碱基的字母都被分离开,每一个字母都将成为数据中的标量值。然后一个接一个的查看数组元素(每一个都是一个单独的字符),这样你就可以进行计数了。这与\autoref{sect:section5.3.2}中的\verb|join|函数是相反的,它把数组中的字符串合并成单个的标量值。(在把字符串拆解成数组后,如果你愿意,可以使用\verb|join|再把这个数组合并成和原来一模一样的字符串。)

在刚才的伪代码中,我将加上下面的这些操作指令:从文件中获取DNA,操作文件数据使DNA序列称为单一的字符串。所以,首先,你把包含原始文件数据行的数组合并起来,通过删除空白把它清理干净,直到只有序列保留下来为止,然后,把它拆解成数组。当然,关键的一点就是最后的这个数组就是我们所需要的,使用它可以在循环中方便地进行计数。与包含行、换行符和其他可能的无用的字符的数组不同,这个数组就是包含单个碱基的数组。

\begin{lstlisting}
read in the DNA from a file

join the lines of the file into a single string $DNA

# make an array out of the bases of $DNA
@DNA = explode $DNA

# initialize the counts
count_of_A = 0
count_of_C = 0
count_of_G = 0
count_of_T = 0

for each base in @DNA

  if base is A
    count_of_A = count_of_A + 1
  if base is C
    count_of_C = count_of_C + 1
  if base is G
    count_of_G = count_of_G + 1
  if base is T
    count_of_T = count_of_T + 1
done

print count_of_A, count_of_C, count_of_G, count_of_T
\end{lstlisting}

如前所述,这里的伪代码更加详细一些。通过把DNA字符串拆解成包含单个字符的数组,它提供了一种查看每个碱基的办法。为了保证计数正确,它还把所有的计数都初始化为0。如果你理解了程序中的初始化,就很容易看出具体发生了什么,这可以防止代码中出现某些错误。(然而,这并不是规定。有时,你可能更加偏好直到使用标量时,一直使它的值保持为未定义。)如果你尝试把变量当做数值来使用,比如把它和另一个数字相加,Perl会假定这个未初始化的变量的值为0。但是这种情况下你通常会得到一个警告信息。

我们已经设计好了程序,现在就把它转化成Perl代码。\autoref{exam:example5.4}是一个可以工作的程序。随着本章的学习,你将看到其他完成该任务的方法,它的速度会更快一些,但此处速度并不是我们关注的焦点。

\textbf{例5-4:确定核苷酸的频率}
\lstinputlisting[label=exam:example5.4]{./scripts/example5-4.pl}

为了演示\autoref{exam:example5.4},我创建了下面这个小的DNA文件,把它命名为\textit{small.dna}:

\begin{lstlisting}
AAAAAAAAAAAAAAGGGGGGGTTTTCCCCCCCC
CCCCCGTCGTAGTAAAGTATGCAGTAGCVG
CCCCCCCCCCGGGGGGGGAAAAAAAAAAAAAAATTTTTTAT
AAACG
\end{lstlisting}

你可以使用钟爱的文本编辑器在计算机上键入\textit{small.dna}文件,也可以从本书的网站上下载它。

注意文件中有一个V,这一一个错误。\footnote{DNA序列数据的文件有时会包含N这样的字符,它表示“未知碱基”,以及其他特定的字符。有时你不得不去查看一下数据来源的文档,比如ABI测序仪或GenBank文件等,看看它们使用了那些字符,各代表什么含义。}下面是\autoref{exam:example5.4}的输出:

\begin{lstlisting}
Please type the filename of the DNA sequence data: small.dna
!!!!!!!! Error - I don't recognize this base: V

A = 40
C = 27
G = 24
T = 17
errors = 1
\end{lstlisting}

现在我们来看看程序中出现的新的知识。打开和读取序列数据与前面的程序是一样的。第一个新知识出现在这一行中:

\begin{lstlisting}
@DNA = split( '', $DNA );
\end{lstlisting}

它的注释解释说它会把\verb|$DNA|拆解成包含单个字符的\verb|@DNA|数组。

\textit{split}与\textit{join}是相对的,最好花些时间去看看这两个命令的文档。使用\textit{split}时,如果第一个参数是空字符串,将会把字符串拆解成单个的字符,而这正是我们所需要的。\footnote{就像你在\textit{split}函数的文档中所看到的那样,第一个参数可以使任何的正则表达式,比如\verb|/\s+/|(一个或多个相邻的空白字符)。}

接下来,有五个标量变量被初始化为了0,就是\verb|$count_of_A|等变量。所谓\textit{初始化}就是给它赋一个初始值,在这个例子中,就是0。

\autoref{exam:example5.4}阐明了\textit{类型}和\textit{初始化}的概念。一个变量的类型决定了它可以存储的数据类型,比如,字符串或数字。到现在为止,我们使用\verb|$DNA|这样的标量变量来存储由A、C、G和T字母构成的字符串。\autoref{exam:example5.4}演示了你也可以用变量变量来存储数字。比如,\verb|$count_of_A|变量就记录着字符A的计数。

标量变量可以存储整数(0、1、-1、2、-2、……),如6.544这样的小数或浮点数,如6.544E6这样用科学计数法表示的数字,它可以转换成$6.544 \times 10^6$或者6,544,000。(\autoref{chap:chapterab}中有更多关于数字类型的介绍。)

在\autoref{exam:example5.4}中,从\verb|$count_of_A|到\verb|$count_of_T|的变量都被初始化为0。\textit{初始化}一个变量意味着在声明该变量后给它一个值。如果你不初始化你的变量,它的值将被假定为\verb|'undef'|。在Perl中,对于一个未定义的变量,如果在数字上下文中使用时它的值为0,在字符串操作中使用时它就是空字符串。尽管Perl程序员通常不去初始化变量,但对于其他的语言来说这确实一个关键的步骤。比如,在C中,未初始化的变量就会有不可预料的值,这可能会导致莫名其妙的输出。你应当养成初始化变量的习惯,这会使得程序更加容易阅读和维护,这非常重要。

所谓\textit{声明}变量就是指定变量的名字及其它属性,如初始值和作用域(对于作用域,请参看\autoref{chap:chapter6}以及\verb|my|变量的讨论)。许多语言要求你在使用变量之前先声明它们。对于本书来说,到目前为止,声明还并不是必需的。从下一章开始将需要声明。在Perl中,除了初始化变量的值外,你还可以声明一个变量的作用域(参看\autoref{chap:chapter6}以及\verb|my|变量的讨论)。许多语言还要求你声明变量的类型,比如“整数”或“字符串”,但Perl并不要求你这么做。

Perl在判断标量变量是什么时非常智能。比如,你可以把数字\verb|1234|(没有引号)赋值给变量,也可以把字符串\verb|'1234'|(有引号)赋值给它。当打印输出时,Perl把变量当成字符串来处理,而当在算术运算中使用时Perl则会把它看做数字,所有这些你都无需担心。\autoref{exam:example5.5}演示了Perl的这种能力。换句话说,在指定变量的数据类型时,Perl并不是很严格。

\textbf{例5-5:演示Perl对数字和字符串的智能化处理}
\lstinputlisting[label=exam:example5.5]{./scripts/example5-5.pl}

\autoref{exam:example5.5}的输出:

\begin{lstlisting}
1234 1234
2468
12341234
\end{lstlisting}

\autoref{exam:example5.5}演示了Perl对标量变量数据类型的智能化处理,它是字符串还是数字,你是要像数字那样对它进行加减,还是像字符串那样把它串联起来。Perl会具体问题具体分析,这使得程序员的工作更加简单一些。Perl为你“作对的事情”。

加下来就是一种循环,\verb|foreach|循环。该循环作用于数组的元素。这一行:

\begin{lstlisting}
foreach $base (@DNA) {
\end{lstlisting}

对\verb|@DNA|数组的元素进行循环处理,每循环一次,标量变量\verb|$base|(或者你起的任意一个名字)就会被设定成数组中的下一个元素。

循环的主体检查每一个碱基,当它发现某个碱基时就会增加它的计数。在Perl中,有四种方法给一个数字加\verb|1|。在这个例子中,你把\verb|++|放在变量的前面,就像这样:

\begin{lstlisting}
++$count; 
\end{lstlisting}

你也可以把\verb|++|放在变量的后面:

\begin{lstlisting}
$count++;
\end{lstlisting}

你可以把它写成这样,加法和赋值的组合:

\begin{lstlisting}
$count = $count + 1;
\end{lstlisting}

或者,作为它的简写,也可以这样:

\begin{lstlisting}
$count += 1;
\end{lstlisting}

这几乎成了选择的烦恼。\verb|++|这种写法对于增加计数来说比较方便,就像我们此处这样。\verb|+=|这种写法节省一些输入,在加除1以外的数字时比较常用。

\autoref{exam:example5.5}中的\verb|foreach|循环也可以写成这样:

\begin{lstlisting}
foreach (@DNA) {
  
  if ( /A/ ) {
    ++$count_of_A;
  } elsif ( /C/  ) {
    ++$count_of_C;
  } elsif ( /G/  ) {
    ++$count_of_G;
  } elsif ( /T/  ) {
    ++$count_of_T;
  } else {
    print "!!!!!!!! Error - I don\'t recognize this base: ";
    print;
    print "\n";
    ++$errors;
  }
}
\end{lstlisting}

\verb|foreach|循环的这种写法:

\begin{lstlisting}
foreach(@DNA) {.
\end{lstlisting}

没有标量值。在\verb|foreach|循环中,如果你不指明存储从数组中读取的标量的变量(在\autoref{exam:example5.5}的循环中担当该职责的是\verb|$base|),Perl就会使用特殊变量\verb|$_|。

此外,在不提供参数的情况下,Perl的许多内置函数都会对这个特殊变量进行操作。在这个例子中,条件测试是简单的模式,Perl假定你对\verb|$_|变量进行模式匹配,所以这和你使用\verb|$_ =~ /A/|是完全一样的。最后,在错误信息中,\verb|print;|语句打印出\verb|$_|变量的值。 

尽管在本书中,我不会过多的使用它,但这个不需要命名的特殊变量\verb|$_|会出现在许多Perl程序中。

\section{操作字符串}
要查看每一个字符,起始也没必要把字符串拆解成数组。事实上,有时你会想避免那种做法。一个大的字符串会占用你计算机的大量内存,对于大的数组也是如此。当你把字符串拆解成数组时,原始的字符串仍然存在着,而你还要对每一个字符制作一个拷贝,作为元素存储到创建的数组中。如果你有一个大的字符串,它已经占用了一大部分可用内存,创建另一个数组可能会导致内存不足。当内存不足时,计算机的性能会大打折扣,它会慢的像乌龟一样,甚至崩溃或冻结(“挂起”)。到现在为止,这些都还不是什么令人担忧的因素,但当你处理大的数据集(比如人类基因组)时,你将不得不考虑这些问题。

那么假设你不想制作一个DNA序列数据的拷贝存进另一个变量。有没有什么方法可以直接查看\verb|$DNA|字符串并对碱基进行计数吗?答案是肯定的。下面是伪代码,紧随其后的是Perl程序:

\begin{lstlisting}
read in the DNA from a file

join the lines of the file into a single string of $DNA

# initialize the counts
count_of_A = 0
count_of_C = 0
count_of_G = 0
count_of_T = 0

for each base at each position in $DNA

  if base is A
    count_of_A = count_of_A + 1
  if base is C
    count_of_C = count_of_C + 1
  if base is G
    count_of_G = count_of_G + 1
  if base is T
    count_of_T = count_of_T + 1
done

print count_of_A, count_of_C, count_of_G, count_of_T
\end{lstlisting}

\autoref{exam:example5.6}是查看DNA字符串中每个碱基的程序。

\textbf{例5-6:确定核苷酸频率,第二次尝试}
\lstinputlisting[label=exam:example5.6]{./scripts/example5-6.pl}

下面是\autoref{exam:example5.6}的输出:

\begin{lstlisting}
Please type the filename of the DNA sequence data: small.dna
!!!!!!!! Error - I don't recognize this vase: V
A = 40
C = 27
G = 24
T = 17
errors = 1
\end{lstlisting}

在\autoref{exam:example5.6}中,我添加了一行代码,来看看文件是否存在:

\begin{lstlisting}
unless ( -e $dna_filename ) {
\end{lstlisting}

有许多各种各样的文件测试操作符,可以参看\autoref{chap:chapterab}或\textit{-X}的Perl文档。注意文件有许多属性,比如大小、权限、文件系统中的位置、文件类型等,对于许多这样的属性都可以简单地使用文件测试操作符进行检测。

此外,注意我保留了关于正则表达式的详细注释,因为正则表达式难于阅读,此处的少许注释可以帮助读者跳过代码。

所有的代码都非常熟悉,除了\verb|for|循环,需要对它进行一些解释:

\begin{lstlisting}
for ( $position = 0 ; $position < length $DNA ; ++$position ) {
  
  # the statements in the block
}
\end{lstlisting}

这里的\textit{for}循环和下面的\textit{while}循环是完全等价的:

\begin{lstlisting}
$position = 0;

while( $position < length $DNA ) {

  # the same statements in the block, plus ...

  ++$position;
}
\end{lstlisting}

花点时间,把这两个循环比较一下。你会看到同样的语句,但是出现在了不同的地方。

如你所见,在\textit{for}循环的循环语句中引入了计数器(\verb|$position|)的初始化和增量,而在\textit{while}循环中,它们则是单独的语句。在\textit{for}循环中,初始化和增量语句都包裹在了括号中,而在\textit{while}循环的括号中只有条件测试。在\textit{for}循环中,你把初始化语句放在了第一个分号的前面,把增量语句放在了第二个分号的后面。初始化语句在循环开始之前就执行了,而且只初始化一次,而增量语句则是在每次循环迭代完成后、条件测试之前执行的。如此所述,它实际上就是完全等价于\textit{while}循环的一种简写方法。

条件测试检测处理到达字符串的位置是否小于字符串的长度。它使用的是Perl的\textit{length}函数。显然,你不能检测超过字符串长度的字符。但对于字符串和数组位置的索引还要多说几句。

默认情况下,Perl假定字符串的索引起始于\verb|0|,它的最后一个字符的索引要比字符串的长度小一。问什么不使用起始于1、终止于字符串长度的索引,而是用这样的索引呢?这其中有很多原因,但它们都有些深奥;可以参看一下关于启蒙(enlightenment)的文档。稍感欣慰的是,许多其他编程语言使用的也是Perl的这种索引方式。(当然,也有编程语言选择了直观的索引方式,从1开始进行索引。好吧。)

这种索引方式对于生物学家来说非常重要,因为他们已经习惯了从1开始对序列进行索引,而不是像Perl这样从\verb|0|开始。在打印输出结果之前,你有时需要把位置索引加1,这样对于非程序员来说就比较容易理解了。这可能稍微有些烦人,但你会慢慢习惯的。

对于数组元素的索引也是一样的,数组的第一个元素的索引是\verb|0|,最后一个元素的索引是\verb|$length-1|。

不管怎么说,当\verb|$position|的值小于等于\verb|length-1|时,条件测试的结果为\verb|真|,当\verb|$position|的值和字符串的长度相等时,测试就会失败。举个例子,假设你有一个包含“seeing”文本的字符串,它的长度为6个字符。其中,“s”的位置索引为0,“g”的位置索引为5,比字符串的长度6小一。

在回来看看代码块,你使用\textit{substr}函数来查看字符串:

\begin{lstlisting}
$base = substr($DNA, $position, 1);
\end{lstlisting}

对于处理字符串来说,这是相当常用的一个函数,你还可以用它来进行插入或删除。在这个例子中,你只想查看一个字符,所以你对字符串\verb|$DNA|使用了\textit{substr}函数,让它找到位置索引为\verb|$position|的那个字符,并把结果保存到标量变量\verb|$base|中。然后,就像前面的\autoref{exam:example5.4}程序那样,对碱基进行计数。

\section{写入文件}
  \autoref{exam:example5.7}演示了对DNA字符串中的核苷酸进行计数的另一种方法。它使用了一个Perl的技巧,这种技巧就是专门为类似的工作设计的。在\textit{while}循环的测试中,它进行了一个全局的正则表达式查找,就像你将看到的,这是一种对字符串中字符进行计数的简洁的方法。

Perl比较好的一点就是,对于那些相当常见的事情,它很可能提供了一种相对简洁的处理方法。(而它的弊端就是对于Perl,有大量知识需要你去学习。)

\autoref{exam:example5.7}的结果,不仅会输出打印到屏幕,还会写入到文件中。实现写入文件的代码如下:

\begin{lstlisting}
# Also write the results to a file called "countbase"

$outputfile = "countbase";

unless ( open(COUNTBASE, ">$outputfile")  ) {

  print "Cannot open file \"$outputfile\" to write to!!\n\n";
  exit;
}

print COUNTBASE "A=$a C=$c G=$g T=$t errors=$e\n";

close(COUNTBASE);
\end{lstlisting}

正如你所见,要写入到文件,你要像读取文件那样调用\textit{open},但有一点不同:在文件名前使用的是大于号\verb|>|。文件句柄成了\verb|print|语句的第一个参数(但其后并没有逗号),这使得\verb|print|语句把它的输出定向到了文件。\footnote{在这种情况下,如果文件已经存在了,它将先被清空然后写入。但也是可以指定其他的行为方式的。如前所述,Perl文档中有关于\textit{open}函数的所有细节,可以设定选项来读取、写入文件,以及其他操作。}

\autoref{exam:example5.7}检查DNA字符串中每个碱基的第三个版本的Perl程序。

\textbf{例5-7:确定核苷酸频率,第三次尝试}
\lstinputlisting[label=exam:example5.7]{./scripts/example5-7.pl}

当你运行\autoref{exam:example5.7}时,会看到类似的输出:

\begin{lstlisting}
Please type the filename of the DNA sequence data: small.dna
A=40 C=27 G=24 T=17 errors=1
\end{lstlisting}

在你运行\autoref{exam:example5.7}后,计数碱基的输出文件中包含以下内容:

\begin{lstlisting}
A=40 C=27 G=24 T=17 errors=1
\end{lstlisting}

\textit{while}循环:

\begin{lstlisting}
while($dna =~ /a/ig){$a++} 
\end{lstlisting}

有它自己的条件测试,包裹在括号中,这是一个字符串匹配的表达式:

\begin{lstlisting}
$dna =~ /a/ig 
\end{lstlisting}

这个表达式查找正则表达式\verb|/a/|,也就是\verb|a|这个字母。既然它使用了\verb|i|修饰符,那这就是不区分大小写的匹配,它将匹配\verb|a|或者\verb|A|。它还使用了全局修饰符,这意味它将匹配字符串中的所有\verb|a|。(没有全局修饰符时,在每次循环迭代时,如果在\verb|$dna|中有“a”,它只会持续返回\verb|真|。)

现在,\verb|while|循环中的这个字符串匹配表达式,使得\textit{while}循环在每一次匹配正则表达式时都执行它的代码块。所以,附加了这只有一个语句的代码块:

\begin{lstlisting}
{$a++}
\end{lstlisting}

在每次匹配正则表达式时递增计数器。换句话说,你会对所有的\verb|a|进行计数。

对于这第三个版本的程序来说,还有一点需要提及一下。你会注意到其中有些语句发生了变化,被缩短了。有些变量的名字更短了,有些语句放在了一行中,同时末尾的\textit{print}语句也更加简洁了。这些都只是书写的另一种方式而已。在你编程时,你将会发现自己会遇到不同的书写方式:适当的去尝试一下吧。

在第三个版本的程序中,计数碱基的方式比较灵活。比如,不需要单独指定各个碱基,你就可以对ACGT以外的所有碱基进行计数。在后面的章节中,你将使用这样的\verb|while|循环来取得良好的效果。然后,还有一种更快的计数碱基的方法。你可以使用\autoref{chap:chapter4}中提到的\textit{tr}转换函数,它的速度更快一些,如果你有大量的DNA需要计数,它将会非常有用:

\begin{lstlisting}
$a = ($dna =~ tr/Aa//);
$c = ($dna =~ tr/Cc//);
$g = ($dna =~ tr/Gg//);
$t = ($dna =~ tr/Tt//);
\end{lstlisting}

\textit{tr}函数返回它在字符串中找到的特定字符的数目,并且,如果替换的字符集为空的话,原始的字符串并不会被改变。这使得它称为一个很好的字符计数器。注意使用\textit{tr}时,你需要同时指定大小写字母。此外,因为\textit{tr}并不接受字符组,所以没法直接对非碱基的字符进行计数。但是,你可以这样做:

\begin{lstlisting}
$basecount = ($dna = ~ tr/ACGTacgt//);
$nonbase = (length $dna) - $basecount)
\end{lstlisting}

使用\textit{tr}的这个程序要比\autoref{exam:example5.7}中使用\textit{while}循环的程序运行快。

你可能会觉得计数碱基的程序有三个(实际上是四个)版本有点多了,尤其是每个版本中的大部分代码都是一样的。程序中唯一改变的就是计数碱基的部分。有没有什么办法,能够让我们方便的只改变计数碱基的部分呢?在\autoref{chap:chapter6}中,你将看到这样的方法,使用子程序来把程序进行分割。

\section{练习题}
\textcolor{red}{\textit{习题5.1}}
\begin{adjustwidth}{1cm}{}
使用循环写一个永不挂起的程序。每次迭代时,条件测试都应该永远为\verb|真|。注意,有的系统会注意到你在一个无穷的循环中,从而自动结束程序。依据你使用的操作系统的不同,结束程序的方法也会有所不同。在Unix和Linux、Windows
MS-DOS命令窗口或者MacOS X shell窗口中,使用Ctrl-C就可以结束程序。
\end{adjustwidth}

\textcolor{red}{\textit{习题5.2}}
\begin{adjustwidth}{1cm}{}
提示用户输入两个(短的)DNA字符串。通过使用\verb|.=|赋值操作符,把第二个附加到第一个DNA字符串的后面,从而把两者串联起来。先打印输出串联后的两个字符串,然后打印输出第二个字符串,但要与串联后字符串末尾对应的字符串对齐。举个例子,如果输入的字符串是\verb|AAAA|和\verb|TTTT|,则要打印输出:
\begin{verbatim}
AAAATTTT
    TTTT
\end{verbatim}
\end{adjustwidth}

\textcolor{red}{\textit{习题5.3}}
\begin{adjustwidth}{1cm}{}
编写一个程序,输出从1到100的所有数字。当然,你的程序代码应该远远小于100行。
\end{adjustwidth}

\textcolor{red}{\textit{习题5.4}}
\begin{adjustwidth}{1cm}{}
编写一个程序,计算DNA链的反向互补链。不要使用\textit{s///}或者\textit{tr}函数。使用\textit{substr}函数,当你计算反向互补链时,要一次检查原始DNA中的一个碱基。(提示:你可能会发现检查原始DNA字符串时,尽管从左到右也是可以的,但从右向左要更加方便一些) 
\end{adjustwidth}

\textcolor{red}{\textit{习题5.5}}
\begin{adjustwidth}{1cm}{}
编写一个程序,报告蛋白质序列中疏水氨基酸的百分比。(想要知道哪些氨基酸是疏水性的,可以参看任何关于蛋白质、分子生物学或细胞生物学导论性的介绍。在\autoref{chap:chapteraa}中你会找到一些有用的资源。)
\end{adjustwidth}

\textcolor{red}{\textit{习题5.6}}
\begin{adjustwidth}{1cm}{}
编写一个程序,检查一下作为参数提供的两个DNA字符串是不是反向互补的。使用Perl的内置函数\textit{split}、\textit{pop}、\textit{shift}和\textit{eq}(\textit{eq}实际上是一个操作符)。
\end{adjustwidth}

\textcolor{red}{\textit{习题5.7}}
\begin{adjustwidth}{1cm}{}
编写一个程序,报告序列的GC含量。(换言之,就是给出DNA中G和C的百分比)。
\end{adjustwidth}

\textcolor{red}{\textit{习题5.8}}
\begin{adjustwidth}{1cm}{}
修改\autoref{exam:example5.3},使它不仅能通过正则表达式找到基序,还能打印输出它找到的基序。举个例子,如果你使用正则表达式查找基序EE.*EE,你的程序应该输出EETVKNDEE。你可以使用特殊变量\verb|$&|。在一次成功的模式匹配后,这个特殊变量会保存匹配到的模式。
\end{adjustwidth}

\textcolor{red}{\textit{习题5.9}}
\begin{adjustwidth}{1cm}{}
编写一个程序,交换DNA字符串中特定位置的两个碱基。(提示:你可以使用Perl的\verb|substr|函数或\verb|slice|函数。)
\end{adjustwidth}

\textcolor{red}{\textit{习题5.10}}
\begin{adjustwidth}{1cm}{}
编写一个程序,写入一个临时文件然后把它删除掉。\textit{unlink}函数可以删除一个文件,举个例子,这样来使用:
\begin{lstlisting}
unlink "tmpfile";
\end{lstlisting}
但要检查一下看看\verb|unlink|是否成功了。
\end{adjustwidth}
